\sloppy

%!!!
% Кроме этого в 00-title.tex надо ручками прописать сам текст задания
%!!!

% Даты начала и конца работы
\renewcommand{\startDay}			{03}
\renewcommand{\startMonth}			{июня}
\renewcommand{\startYear}			{2024}
\renewcommand{\finalDay}			{03}
\renewcommand{\finalMonth}  		{июля}
\renewcommand{\finalYear}			{2024}
\renewcommand{\finalDate}			{<<\finalDay>> \finalMonth\ \finalYear г.}
% Сокращения официальных названий
\renewcommand{\schoolNameFull}		{Институт опережающих технологий <<Школа Икс>>}
\renewcommand{\schoolName}			{ДГТУ ИОТ <<Школа Икс>>}
\renewcommand{\orgNameFull}			{
	ФЕДЕРАЛЬНОЕ ГОСУДАРСТВЕННОЕ БЮДЖЕТНОЕ
	ОБРАЗОВАТЕЛЬНОЕ УЧРЕЖДЕНИЕ ВЫСШЕГО ОБРАЗОВАНИЯ
	«ДОНСКОЙ ГОСУДАРСТВЕННЫЙ ТЕХНИЧЕСКИЙ УНИВЕРСИТЕТ»
	}
\renewcommand{\orgName}				{ФГБОУ ВО <<ДГТУ>>}
% Личная информация
\renewcommand{\facKey}				{15.03.06} % номер направления подготовки
\renewcommand{\facName}				{Мехатроника и робототехника} % название направления подготовки
\renewcommand{\decim}				{\facKey.930000.000} % Для рамок, поищи в DSTUEskd.sty
\renewcommand{\developer}			{Лемешкин}
\renewcommand{\developerIOF}		{Р. А. \developer}
\renewcommand{\developerFaImOt}		{\developer Руслан Андреевич}
\renewcommand{\developerYear}		{3} % курс обучения (скорее всего ты его будешь менять чаще всего ;)
\renewcommand{\developerGroup}		{ХР\developerYear1}
% Названия конкретного документа
\renewcommand{\docTypeFull}			{Пояснительная записка} % Тип документа (тоже в DSTUEskd.sty ищи)
\renewcommand{\docType}				{ПЗ}
\renewcommand{\module}				{Умный дом} % Вид практики или название дисциплины курсача
\renewcommand{\internOrg}			{\schoolName} % Название предприятия практики
\renewcommand{\devNameFull}			{\module} % Название темы для курсача
\renewcommand{\devName}				{УД} % наверное только для рамок
% И преподы
\renewcommand{\schTutorPos}			{к.т.н., доц.} % От уника чел для практики или владелец
\renewcommand{\schTutorIOF}			{А.В. Авилов}%  модуля / наставник для курсача и рамок
\renewcommand{\baseTutorPos}		{директор} % От предприятия чел для практики
\renewcommand{\baseTutorIOF}		{П.В. Герасин} % или утвердивший для курсача (директор или ректор)
\renewcommand{\nControl}			{Петров} % смотри в файле с рамками, мож и надо...

\DSTUHaveLRIfalse		 % Вставить лист регистрации изменений
\DSTUHaveLYfalse		% Вставить лист утверждения
\DSTUHaveESKDFrametrue   % Использовать рамку

\usepackage{pdflscape}
% \usepackage{lscape}

%\makeatletter
%\newcommand{\keepwithnext}{\@beginparpenalty 10000}
%\makeatother