\section*{Введение}
\subsection*{Цель работы:}
Ознакомиться c принципами работы энкодеров JA12-N20B 6В и изучить их.
\subsection*{Используемые приборы и оборудование:}
\begin{itemize}
    \item ноутбук;
    \item мотор-редуктор с энкодером JA12-N20B 6В;
    \item Arduino Mega;
    \item макетная плата и соединительные провода.
\end{itemize}

\section{Ход выполнения работы:}
Перед началом работы мы изучили принцип работы энкодера, а также эффект холла.

\subsection{Принцип работы энкодера в моторе-редукторе JA12-N20B}
Энкодер --- электротехническое устройство, предоставляющее возможность определять направление вращение вала мотора и скорость его вращения. Применение энкодера в конструкции мотора-редуктора позволяет определить момент начала вращения двигателя, а также рассчитать пройденное расстояние, если изделие установлено в самодвижущиеся платформы.

В моторе-редукторе JA12-N20B энкодер основан на эффекте Холла, который заключается в возникновении разности потенциалов в проводнике, по которому течёт электрический ток и находящегося внутри магнитного поля.

В моторе-редукторе JA12-N20B энкодер состоит из цилиндрического магнита и двух однополярных цифровых датчиков магнитного поля (датчиков Холла) с маркировкой 44Е, размещённых на валу мотора.

При вращении вала вместе с ним вращается и магнит, что вызывает изменение магнитного поля. Это изменение приводит к возникновению напряжения в датчиках Холла, которые передают цифровой сигнал на микроконтроллер.

Направление вращения определяется так: два датчика расположены с фазовым сдвигом (обычно 90 градусов). Если сигнал от первого датчика опережает сигнал от второго, это указывает на одно направление вращения и наоборот.

Таким образом, энкодер в мотор-редукторе JA12-N20B фиксирует факт вращения вала и определяет направление его вращения.

\subsection{Практическая часть}
После того, как мы разобрались с принципом работы датчика, мы собрали на макетной плате схему и написали код для вывода данных с энкодера.

Далее мы поставили мотор на стол так, чтобы колесо свисало со стола и медленно провернули колесо на полный оборот и замерили количество тиков энкодера. Так мы узнали, что один оборот колеса равняется 349 тикам. Также мы узнали, что если вращать колесо слишком быстро, то энкодер проскальзывает и фиксирует не все тики.

После этого мы аккуратно провернули на один оборот магнит энкодера и узнали, что один оборот двигателя равен семи тикам.

\section*{Вывод:}
Энкодер может проскальзывать при высокой скорости, которая может возникать в том числе и при спуске робота по наклонной плоскости, а значит желательно ставить датчик напрямую на ось колеса. При этом коэффициент редукции JA12-N20B действительно равен чётко 50.
